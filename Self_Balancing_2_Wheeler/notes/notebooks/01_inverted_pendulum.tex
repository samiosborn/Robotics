% 01_inverted_pendulum.tex
\documentclass[11pt,a4paper]{article}

\usepackage{amsmath, amssymb, amsfonts}
\usepackage{physics}
\usepackage{bm}
\usepackage{graphicx}
\usepackage{geometry}
\geometry{margin=1in}

\title{Inverted Pendulum Model for a Self-Balancing Robot}
\date{}

\begin{document}
\maketitle

\section{Physical Model}

Consider a two-wheel self-balancing robot. 
The body is modelled as a rigid rod of mass \(m\) and centre-of-mass at distance \(l\) above the wheel axle. 
Let
\[
\theta(t)
\]
denote the tilt angle of the body, measured from the upright position, with \(\theta>0\) meaning the robot is falling forwards.

We approximate the robot as an inverted pendulum pivoting at the wheel axle. For now, we ignore wheel translation and treat the axle as a fixed pivot. 
This is a simplification but sufficient for understanding the balance controller.

\section{Equations of Motion}

Using a small-angle approximation (\(\sin \theta \approx \theta\), \(\cos \theta \approx 1\)) the rotational dynamics about the axle are
\[
I \ddot{\theta} = m g l \, \theta + \tau
\]
where
\begin{itemize}
  \item \(I\) is the moment of inertia of the body about the axle
  \item \(g\) is gravitational acceleration
  \item \(\tau\) is the control torque applied by the motors about the axle
\end{itemize}

Rewriting,
\[
\ddot{\theta} = \frac{m g l}{I}\,\theta + \frac{1}{I}\,\tau
\]

This is an unstable linear system: if \(\tau = 0\) and \(\theta \neq 0\), the solution diverges exponentially.

\section{State Definition}

For balance control, a minimal state is
\[
x(t) = \begin{bmatrix} \theta(t) \\ \dot{\theta}(t) \end{bmatrix}
\]
Then the continuous-time dynamics can be written as
\[
\dot{x}(t) =
\begin{bmatrix}
\dot{\theta}(t) \\
\ddot{\theta}(t)
\end{bmatrix}
=
\begin{bmatrix}
\dot{\theta}(t) \\
\frac{m g l}{I}\,\theta(t) + \frac{1}{I}\,\tau(t)
\end{bmatrix}
\]

Thus, the state is fully determined by tilt angle and angular velocity. Higher derivatives are determined by these and the input torque.

\section{Complementary Filter}

The IMU provides:
\begin{itemize}
  \item Accelerometer: measures specific force, from which we can estimate tilt angle via the direction of gravity.
  \item Gyroscope: measures angular velocity \(\dot{\theta}\)
\end{itemize}

Let
\[
\theta_{\text{acc}}(t)
\]
be the tilt estimate from the accelerometer and
\[
\dot{\theta}_{\text{gyro}}(t)
\]
be the angular velocity from the gyroscope.

A simple discrete-time complementary filter with sample period \(\Delta t\) is
\[
\theta_{k}^{\text{est}} = \alpha \left( \theta_{k-1}^{\text{est}} + \dot{\theta}_{k}^{\text{gyro}} \Delta t \right)
+ (1-\alpha)\,\theta_{k}^{\text{acc}}
\]
with \(0 < \alpha < 1\)

Here:
\begin{itemize}
  \item The gyroscope term \(\theta_{k-1}^{\text{est}} + \dot{\theta}_{k}^{\text{gyro}} \Delta t\) is a high-pass path (good at short-term changes, but drifts).
  \item The accelerometer term \(\theta_{k}^{\text{acc}}\) is a low-pass path (no long-term drift, but noisy under acceleration).
\end{itemize}

\section{PID Balance Control}

Define the error
\[
e(t) = \theta_{\text{ref}} - \theta_{\text{est}}(t)
\]
where \(\theta_{\text{ref}} = 0\) is upright.

A continuous-time PID controller is
\[
u(t) = K_P e(t) + K_I \int_0^t e(\tau)\,d\tau + K_D \frac{d e(t)}{dt}
\]

In discrete time with time step \(\Delta t\) and samples indexed by \(k\),
\begin{align*}
e_k &= \theta_{\text{ref}} - \theta_{\text{est},k}, \\
I_k &= I_{k-1} + e_k \Delta t, \\
D_k &= \frac{e_k - e_{k-1}}{\Delta t}
\end{align*}
and the control command is
\[
u_k = K_P e_k + K_I I_k + K_D D_k
\]

In our robot, \(u_k\) will be mapped to a motor command (e.g. PWM duty cycle or velocity setpoint).

\end{document}
