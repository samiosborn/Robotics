\documentclass[11pt,a4paper]{article}

\usepackage[margin=1in]{geometry}
\usepackage{amsmath, amssymb, amsfonts, bm}
\usepackage{graphicx}
\usepackage{physics}
\usepackage{hyperref}
\usepackage{enumitem}
\usepackage{titlesec}

\title{SLAM Geometry Notes}
\author{Sami Osborn}
\date{}

\setlength{\parindent}{0pt}
\setlength{\parskip}{6pt}

\begin{document}

\maketitle

\section{Coordinate Frames}

A coordinate frame is the combination of a defined origin point and a set of three orthogonal unit axes directions.

\subsection*{World / Global Frame (W)}
The world frame (or global frame) is fixed and is generally where the object starts.

\subsection*{Camera / Body Frame (C/B)}
The camera frame (or body frame) is where the camera is in space. It moves with the robot.

\subsection*{Landmark Frame (L)}
This is used for local feature points.

\section{Transformations}

We define \( T_{AB} \) as the transformation from frame \( B \) to \( A \). A pose is a pair \( (R, t) \) describing the orientation and position of one frame relative to another.

\subsection*{Definition of a Lie Group}

Formally, a Lie group \( G \) is a set that satisfies:

\begin{enumerate}[label=\arabic*.]
    \item \textbf{Group:}
    There exists a binary operation (group multiplication)
    \(\circ : G \times G \rightarrow G\)
    such that:
    \begin{itemize}
        \item \textbf{Closure:} \( a \circ b \in G \)
        \item \textbf{Identity element:} \( e \in G \) such that \( e \circ g = g \circ e = g \)
        \item \textbf{Inverse element:} for all \( g \in G \), there exists \( g^{-1} \in G \)
        \item \textbf{Associativity:} for all \( a,b,c \in G \), \( (a \circ b) \circ c = a \circ (b \circ c) \)
    \end{itemize}

    \item \textbf{Smooth manifold structure:}
    The set \( G \) is also a differentiable manifold.

    \item \textbf{Smooth group operations:}
    The group operations are smooth (infinitely differentiable) maps:
    \[
    \begin{aligned}
    \text{Multiplication:} &\quad G \times G \to G, \quad (g_1, g_2) \mapsto g_1 \circ g_2 \\
    \text{Inverse:} &\quad G \to G, \quad g \mapsto g^{-1}
    \end{aligned}
    \]
\end{enumerate}

\subsection*{Switching Coordinate Frames}

To move between frames, we first translate the origin, then rotate the frame. Consider moving from world to camera frame:
\[
p_C = R_{CW} (p_W - t_{CW})
\]
\[
p_W = R_{WC} p_C + t_{CW}
\]
Where:
\begin{itemize}
    \item \( R_{WC} = R_{CW}^T \)
\end{itemize}

\subsection*{Rotation Groups -- SO(3)}

The Special Orthogonal Group of degree 3, \( SO(3) \), is the group of rotations in 3D.

\subsection*{Rotation Matrix, \( R \)}
A rotation matrix \( R \in \mathbb{R}^{3 \times 3} \) satisfies:
\[
R R^T = I, \quad \det(R) = 1
\]

\subsection*{Rotation Axis--Angle Representation}

Any 3D rotation can be expressed as a rotation by an angle \( \theta \) about a unit vector (axis) \( \hat{u} \in \mathbb{R}^3 \):
\[
(\hat{u}, \theta) \quad \text{with } \| \hat{u} \| = 1
\]

\subsection*{Axis--Angle to a Rotation Matrix}

The corresponding rotation matrix is obtained using the Rodrigues formula:
\[
R(\hat{u}, \theta) = I + \sin\theta \, \hat{u}^\wedge + (1 - \cos\theta)(\hat{u}^\wedge)^2
\]

Where the hat operator \( (\cdot)^\wedge \) maps a 3D vector to its skew-symmetric matrix:
\[
\hat{u}^\wedge =
\begin{bmatrix}
        u_x \\
        u_y \\
        u_z
\end{bmatrix}^\wedge =
\begin{bmatrix}
0 & -u_z & u_y \\
u_z & 0 & -u_x \\
-u_y & u_x & 0
\end{bmatrix}
\]

\paragraph{Applying to a vector: } Let \( \mathbf{v} \in \mathbb{R}^3 \)

\[
\mathbf{v}' = R(\hat{u}, \theta) \mathbf{v}
\]

Expanded as Rodrigues' formula:
\[
\mathbf{v}' = \mathbf{v} \cos(\theta) + (\hat{u} \times \mathbf{v}) \sin(\theta) + \hat{u} (\hat{u} \cdot \mathbf{v})(1-\cos(\theta))
\]

\subsection*{Extracting Axis--Angle from a Rotation Matrix}

Given a rotation matrix \( R \in SO(3) \):

\[
\theta = \cos^{-1}\!\left( \frac{\mathrm{trace}(R) - 1}{2} \right)
\]

If \( \theta \neq 0 \), the rotation axis \( \hat{u} \) can be recovered as:
\[
\hat{u} =
\frac{1}{2 \sin\theta}
\begin{bmatrix}
R_{32} - R_{23} \\
R_{13} - R_{31} \\
R_{21} - R_{12}
\end{bmatrix}
\]

\subsection*{Quaternion Vector}

A quaternion vector is:
\[
q = q_w + q_x i + q_y j + q_z k
\]

Where:
\begin{itemize}
    \item \( q_w \) is the scalar component.
    \item \( \mathbf{q_v}\) is the vector component.
\end{itemize}

\paragraph{Pure quaternion: } \( \mathbf{v} = [0, \mathbf{v}] \)

\paragraph{Multiplication: } Consider two quaternions, \( [s_1, \mathbf{a}], [s_2, \mathbf{b}] \):

\[
[s_1, \mathbf{a}][s_2, \mathbf{b}] = [s_1 s_2 - a \cdot b, \quad s_1 \mathbf{b} + s_2 \mathbf{a} + \mathbf{a} \times \mathbf{b}]
\]

\subsection*{Rotation Quaternion}

A unit quaternion \( q = [q_w, q_x, q_y, q_z]^T \in \mathbb{R}^4 \) represents the same rotation, where:
\[
q_w = \cos\left(\frac{\theta}{2}\right), \quad
\mathbf{q}_v =
\begin{bmatrix}
q_x \\ q_y \\ q_z
\end{bmatrix}
= \hat{u} \sin\left(\frac{\theta}{2}\right)
\]
Thus:
\[
\mathbf{q} =
\begin{bmatrix}
\cos\left(\frac{\theta}{2}\right) \\
\hat{u}_x \sin\left(\frac{\theta}{2}\right) \\
\hat{u}_y \sin\left(\frac{\theta}{2}\right) \\
\hat{u}_z \sin\left(\frac{\theta}{2}\right)
\end{bmatrix}
\]
Inversely:
\[
\mathbf{q^{-1}} = \mathbf{q*} =
\begin{bmatrix}
\cos\left(\frac{\theta}{2}\right) \\
- \hat{u}_x \sin\left(\frac{\theta}{2}\right) \\
- \hat{u}_y \sin\left(\frac{\theta}{2}\right) \\
- \hat{u}_z \sin\left(\frac{\theta}{2}\right)
\end{bmatrix}
\]

Where \( u \) is the axis of rotation, and \( \theta \) is the rotation angle. It's more computationally stable and efficient than Euler angles or rotation matrices.

\subsection*{Applying a Rotation Quaternion}

Let \( \mathbf{v} \) be a pure quaternion - i.e. a point in 3D.

To rotate \( v \) by the rotation quaternion \( q \) we perform:
\[
v' = q v q^{-1}
\]

\subsection*{Quaternion to Axis--Angle Conversion}

Given a unit quaternion \( q = [q_w, q_x, q_y, q_z]^T \):

\[
\theta = 2 \cos^{-1}(q_w), \quad
\hat{u} =
\frac{1}{\sin(\theta/2)}
\begin{bmatrix}
q_x \\ q_y \\ q_z
\end{bmatrix}
\quad \text{for } \sin(\tfrac{\theta}{2}) \neq 0
\]

\subsection*{Quaternion to Rotation Matrix}

The rotation matrix corresponding to \( q \) is:
\[
R(q) =
\begin{bmatrix}
1 - 2(q_y^2 + q_z^2) & 2(q_x q_y - q_z q_w) & 2(q_x q_z + q_y q_w) \\
2(q_x q_y + q_z q_w) & 1 - 2(q_x^2 + q_z^2) & 2(q_y q_z - q_x q_w) \\
2(q_x q_z - q_y q_w) & 2(q_y q_z + q_x q_w) & 1 - 2(q_x^2 + q_y^2)
\end{bmatrix}
\]

\subsection*{Rotation Matrix to Quaternion}

Given a rotation matrix,
\[
R =
\begin{bmatrix}
R_{11} & R_{12} & R_{13} \\
R_{21} & R_{22} & R_{23} \\
R_{31} & R_{32} & R_{33}
\end{bmatrix} \in SO(3)
\]
The equivalent quaternion \( q = [q_w, q_x, q_y, q_z]^T \) can be extracted from the matrix elements.

\paragraph{Trace-based formula:} For trace > 0 set: \(t = R_{11} + R_{22} + R_{33} \)

Then:
\[
\begin{cases}
q_w = \tfrac{1}{2}\sqrt{1 + t} \\
q_x = \tfrac{1}{4q_w}(R_{32} - R_{23}) \\
q_y = \tfrac{1}{4q_w}(R_{13} - R_{31}) \\
q_z = \tfrac{1}{4q_w}(R_{21} - R_{12})
\end{cases}
\quad \text{if } t > 0
\]

This formula works well when the trace of \(R\) is positive (small rotation angles).

\paragraph{Numerically stable version:}
For rotations near \(180^\circ\), the trace \(t\) may become small or negative.

To avoid numerical instability, choose the largest diagonal element of \(R\) and compute accordingly:

\[
\text{If } R_{11} \text{ is largest:}
\quad
\begin{cases}
q_x = \tfrac{1}{2}\sqrt{1 + R_{11} - R_{22} - R_{33}} \\
q_y = \tfrac{1}{4q_x}(R_{12} + R_{21}) \\
q_z = \tfrac{1}{4q_x}(R_{13} + R_{31}) \\
q_w = \tfrac{1}{4q_x}(R_{32} - R_{23})
\end{cases}
\]

\[
\text{If } R_{22} \text{ is largest:}
\quad
\begin{cases}
q_y = \tfrac{1}{2}\sqrt{1 - R_{11} + R_{22} - R_{33}} \\
q_x = \tfrac{1}{4q_y}(R_{12} + R_{21}) \\
q_z = \tfrac{1}{4q_y}(R_{23} + R_{32}) \\
q_w = \tfrac{1}{4q_y}(R_{13} - R_{31})
\end{cases}
\]

\[
\text{If } R_{33} \text{ is largest:}
\quad
\begin{cases}
q_z = \tfrac{1}{2}\sqrt{1 - R_{11} - R_{22} + R_{33}} \\
q_x = \tfrac{1}{4q_z}(R_{13} + R_{31}) \\
q_y = \tfrac{1}{4q_z}(R_{23} + R_{32}) \\
q_w = \tfrac{1}{4q_z}(R_{21} - R_{12})
\end{cases}
\]

\paragraph{Normalisation:}
After computing, normalise to ensure the quaternion lies on the unit 4-sphere:
\[
q \leftarrow \frac{q}{\|q\|}
\]

\subsection*{Homogeneous Coordinates and the Special Euclidean Group \( SE(3) \)}

The Special Euclidean group \( SE(3) \) represents all rigid-body transformations (rotations and translations) in 3D space:

\[
T_{WC} =
\begin{bmatrix}
R_{WC} & t_{WC} \\
0 & 1
\end{bmatrix}, \quad
R_{WC} \in SO(3), \; t_{WC} \in \mathbb{R}^3
\]

A 3D point expressed in homogeneous coordinates is written as:

\[
\tilde{p}_W =
\begin{bmatrix}
p_W \\ 1
\end{bmatrix}, \qquad
\tilde{p}_C = T_{CW} \, \tilde{p}_W
\]

\subsection*{World to Camera Transformation}

At time step \( i \), the camera pose in the world frame is
\( T_{WC_i} = (R_{WC_i}, t_{WC_i}) \).
Let \( p_{W_j} \) denote the \( j \)-th landmark (fixed in the world frame).
The coordinates of this landmark in the camera frame are:

\[
p_{C_j} = R_{CW_i} (p_{W_j} - t_{CW_i})
\]

The homogeneous transformation matrix from world to camera is therefore:

\[
T_{CW_i} =
\begin{bmatrix}
R_{CW_i} & t_{WC_i} \\
0 & 1
\end{bmatrix} \in SE(3)
\]

Where: \( t_{WC_i} = -R_{CW_i} t_{CW_i} \)

Its inverse gives the camera-to-world transformation:

\[
T_{WC_i} = T_{CW_i}^{-1} =
\begin{bmatrix}
R_{WC_i} & t_{CW_i} \\
0 & 1
\end{bmatrix}
\]

Where: \( R_{WC_i} = R_{CW_i}^T \)

Thus, a point in the camera frame can be mapped back to world coordinates as:

\[
p_{W_j} = R_{CW_i}^T p_{C_j} + t_{CW_i}
\]

\subsection*{Algebras and Lie Algebras}

\subsection*{Definition of an Algebra}
An algebra is a vector space equipped with a multiplication operation combining two elements of the space to produce another element of the same space.

Formally, an algebra \( \mathcal{A} \) over a field satisfies:
\begin{enumerate}[label=\arabic*.]
    \item \( \mathcal{A} \) is a vector space over the field.
    \item There exists a bilinear product \( \cdot : \mathcal{A} \times \mathcal{A} \to \mathcal{A} \).
    \item For all \( a,b,c \in \mathcal{A} \) and \( \alpha, \beta \in \mathbb{R} \):
    \[
    (\alpha a + \beta b) \cdot c = \alpha (a \cdot c) + \beta (b \cdot c)
    \]
    \[
    a \cdot (\alpha b + \beta c) = \alpha (a \cdot b) + \beta (a \cdot c)
    \]
\end{enumerate}

\subsection*{Definition of a Lie Algebra}
A Lie algebra is an algebra with a Lie bracket operation:
\[
[a, b] = ab - ba
\]
That is: i) bilinear, ii) anti-symmetric, and iii) satisfies the Jacobi identity:
\[
[a,[b,c]] + [b,[c,a]] + [c,[a,b]] = 0
\]

Examples:
\begin{itemize}
    \item SO(3) $\leftrightarrow$ so(3): 3×3 skew-symmetric matrices.
    \item SE(3) $\leftrightarrow$ se(3): 4×4 matrices combining rotation and translation.
\end{itemize}

\subsection*{SE(3) Lie Algebra and the Hat Operator}

A small pose update can be expressed as a 6D vector:
\[
\xi =
\begin{bmatrix}
\rho \\ \phi
\end{bmatrix} \in \mathbb{R}^6
\]
Where: \( \rho \in \mathbb{R}^3 \) (translation), \( \phi \in \mathbb{R}^3 \) (rotation)

\subsection*{Hat Operator \( (\cdot)^\wedge \)}

For rotations:
\[
\phi^\wedge =
\begin{bmatrix}
0 & -\phi_z & \phi_y \\
\phi_z & 0 & -\phi_x \\
-\phi_y & \phi_x & 0
\end{bmatrix}
\]

For rigid-body motion:
\[
\xi^\wedge =
\begin{bmatrix}
\phi^\wedge & \rho \\
0 & 0
\end{bmatrix}
\]

\subsection*{Exponential Map}

The exponential map links the Lie algebra (tangent space) to the Lie group (manifold):

\[
T = \exp(\xi^\wedge) =
\begin{bmatrix}
\exp(\phi^\wedge) & J \rho \\
0 & 1
\end{bmatrix}
\]

For small angles \( \|\phi\| \approx 0 \):
\[
R \approx I + \phi^\wedge, \quad J \approx I + \frac{1}{2}\phi^\wedge
\]

For finite angles:
\[
R = I + \frac{\sin \theta}{\theta}\phi^\wedge + \frac{1 - \cos \theta}{\theta^2}(\phi^\wedge)^2
\]
\[
J = I + \frac{1 - \cos \theta}{\theta^2}\phi^\wedge + \frac{\theta - \sin \theta}{\theta^3}(\phi^\wedge)^2
\]
Where: \( \theta = \|\phi\| \)



\end{document}
