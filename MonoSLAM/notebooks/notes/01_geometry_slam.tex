\documentclass[11pt,a4paper]{article}

\usepackage[margin=1in]{geometry}
\usepackage{amsmath, amssymb, amsfonts, bm}
\usepackage{graphicx}
\usepackage{physics}
\usepackage{hyperref}
\usepackage{enumitem}
\usepackage{titlesec}

\title{SLAM Geometry Notes}
\author{Sami Osborn}
\date{}

\setlength{\parindent}{0pt}
\setlength{\parskip}{6pt}

\begin{document}
\maketitle

% =====================================================
\section{Coordinate Frames}
% =====================================================

A coordinate frame consists of an origin and a set of three orthogonal unit axes.

\subsection*{World / Global Frame (W)}
The world frame (or global frame) is fixed and generally serves as the reference frame.

\subsection*{Camera / Body Frame (C/B)}
The camera or body frame moves with the robot and expresses points relative to the robot’s position.

\subsection*{Landmark Frame (L)}
A local frame used for describing static scene landmarks.

% =====================================================
\section{Transformations and Lie Groups}
% =====================================================

\subsection*{Rigid Transformations}
We denote \(T_{AB}\) as the transformation from frame \(B\) to \(A\).
A pose is a pair \((R, t)\), where \(R\) is the rotation and \(t\) is the translation.

Switching between world and camera coordinates:
\[
p_C = R_{CW} (p_W - t_{CW}), \qquad
p_W = R_{WC} p_C + t_{CW}, \qquad R_{WC} = R_{CW}^T
\]

\subsection*{Definition of a Lie Group}
A Lie group \(G\) is both a group and a smooth manifold with smooth group operations:
\[
(g_1, g_2) \mapsto g_1 g_2, \qquad g \mapsto g^{-1}
\]

\subsection*{Definition of a Lie Algebra}
A Lie algebra is a vector space equipped with a bilinear, antisymmetric bracket
\([a,b] = ab - ba\)
satisfying the Jacobi identity:
\[
[a,[b,c]] + [b,[c,a]] + [c,[a,b]] = 0
\]

Examples:
\[
SO(3) \leftrightarrow \mathfrak{so}(3), \qquad
SE(3) \leftrightarrow \mathfrak{se}(3)
\]
The lowercase symbols denote the Lie algebras (tangent spaces at the identity), while uppercase symbols denote the Lie groups (manifolds of transformations).

% =====================================================
\section{The Special Orthogonal Group \(SO(3)\) }
% =====================================================

\subsection*{Definition}
\[
SO(3) = \{ R \in \mathbb{R}^{3\times3} \mid R^T R = I,\ \det(R)=1 \}
\]
It represents all 3D rotations. Each \(R \in SO(3)\) preserves lengths and angles.

\subsection*{Associated Lie Algebra \(\mathfrak{so}(3)\)}
The Lie algebra of \(SO(3)\) is
\[
\mathfrak{so}(3) = \{ A \in \mathbb{R}^{3\times3} \mid A^T = -A \}
\]
Every element of \(\mathfrak{so}(3)\) corresponds to a vector \(\phi = (\phi_x,\phi_y,\phi_z)^T \in \mathbb{R}^3\) via the \textbf{hat operator}:
\[
\phi^\wedge =
\begin{bmatrix}
0 & -\phi_z & \phi_y \\
\phi_z & 0 & -\phi_x \\
-\phi_y & \phi_x & 0
\end{bmatrix},
\qquad
(\phi^\wedge)^\vee = \phi
\]
The hat operator maps a vector to its corresponding skew-symmetric matrix such that \(\phi^\wedge v = \phi \times v\).

\subsection*{Exponential Map on \(\mathfrak{so}(3)\)}

The exponential map links infinitesimal rotations in the Lie algebra \(\mathfrak{so}(3)\) to finite rotations in the Lie group \(SO(3)\).
For a rotation vector \(\phi \in \mathbb{R}^3\) with skew-symmetric form \(\phi^\wedge \in \mathfrak{so}(3)\),
\[
\exp: \mathfrak{so}(3) \to SO(3), \qquad R(\phi) = \exp(\phi^\wedge)
\]
This expression arises by integrating the differential equation:
\[
\dot{R}(t) = \phi^\wedge R(t), \qquad R(0) = I
\]
whose unique solution is \(R(t) = \exp(t\,\phi^\wedge)\).
The trajectory \(R(t)\) forms a smooth one-parameter subgroup of \(SO(3)\), representing continuous rotation with constant angular velocity \(\phi\).
Evaluating at \(t = 1\) gives the finite rotation \(R(\phi)\), corresponding to a rotation by angle \(\|\phi\|\) about the unit axis \(\hat{u} = \phi / \|\phi\|\).

\subsection*{Rodrigues’ Formula}

Let \(\theta = \|\phi\|\) and \(\hat{u} = \phi / \theta\).
The exponential of a skew-symmetric matrix admits a closed form:
\[
R(\phi) = I + \frac{\sin\theta}{\theta}\phi^\wedge + \frac{1 - \cos\theta}{\theta^2}(\phi^\wedge)^2
\]
which is known as Rodrigues’ rotation formula. Expanding \(\exp(\phi^\wedge)\) via the Taylor series and using \((\hat{u}^\wedge)^3=-\hat{u}^\wedge\) gives the above closed form.

\subsection*{Axis–Angle Representation}
A rotation can equivalently be described by an angle \(\theta\) about a unit axis \(\hat{u}\):
\[
R(\hat{u},\theta) = I + \sin\theta\,\hat{u}^\wedge + (1 - \cos\theta)(\hat{u}^\wedge)^2
\]
and inversely,
\[
\theta = \cos^{-1}\!\left(\frac{\mathrm{trace}(R)-1}{2}\right), \quad
\hat{u} = \frac{1}{2\sin\theta}
\begin{bmatrix}
R_{32}-R_{23}\\R_{13}-R_{31}\\R_{21}-R_{12}
\end{bmatrix}
\]

% =====================================================
\section{Quaternion Representation of Rotation}
% =====================================================

A unit quaternion \(q = [q_w, q_x, q_y, q_z]^T\) encodes rotation as:
\[
q_w = \cos\!\left(\frac{\theta}{2}\right), \quad
\mathbf{q_v} =
\begin{bmatrix}
q_x \\ q_y \\ q_z
\end{bmatrix}
= \hat{u}\sin\!\left(\frac{\theta}{2}\right)
\]
Quaternion conjugate and inverse:
\[
q^* = [q_w, -q_x, -q_y, -q_z]^T, \qquad q^{-1} = q^*
\]

\subsection*{Applying a Quaternion Rotation}
Represent a vector \(v\) as a pure quaternion \([0, \mathbf{v}]\).
The rotated vector is given by:
\[
v' = q v q^{-1}
\]

\subsection*{Quaternion to Matrix and Back}
Rotation matrix from quaternion:
\[
R(q) =
\begin{bmatrix}
1 - 2(q_y^2 + q_z^2) & 2(q_x q_y - q_z q_w) & 2(q_x q_z + q_y q_w) \\
2(q_x q_y + q_z q_w) & 1 - 2(q_x^2 + q_z^2) & 2(q_y q_z - q_x q_w) \\
2(q_x q_z - q_y q_w) & 2(q_y q_z + q_x q_w) & 1 - 2(q_x^2 + q_y^2)
\end{bmatrix}
\]

To recover a quaternion from \(R\):
\[
\theta = 2\cos^{-1}(q_w), \quad
\hat{u} = \frac{1}{\sin(\theta/2)}
\begin{bmatrix}
q_x \\ q_y \\ q_z
\end{bmatrix}, \quad \sin(\tfrac{\theta}{2}) \neq 0
\]

% =====================================================
\section{The Special Euclidean Group \(SE(3)\)}
% =====================================================

\subsection*{Definition}
The group of rigid-body motions in 3D:
\[
SE(3) =
\left\{
\begin{bmatrix}
R & t \\ 0 & 1
\end{bmatrix}
\middle|
R \in SO(3),\ t \in \mathbb{R}^3
\right\}
\]

\subsection*{Homogeneous Coordinates}
A point \(p_W\) in homogeneous form:
\[
\tilde{p}_W =
\begin{bmatrix} p_W \\ 1 \end{bmatrix},
\qquad
\tilde{p}_C = T_{CW}\tilde{p}_W
\]
Inverse transform:
\[
T_{WC} = T_{CW}^{-1} =
\begin{bmatrix}
R^T & -R^T t \\ 0 & 1
\end{bmatrix}
\]

% =====================================================
\section{The Lie Algebra \(\mathfrak{se}(3)\) and Exponential Map}
% =====================================================

\subsection*{Elements and Hat Operator}
An element of \(\mathfrak{se}(3)\) is a 6D twist:
\[
\xi =
\begin{bmatrix}
\rho \\ \phi
\end{bmatrix}, \quad
\rho, \phi \in \mathbb{R}^3
\]
with hat representation:
\[
\xi^\wedge =
\begin{bmatrix}
\phi^\wedge & \rho \\
0 & 0
\end{bmatrix}
\]

\subsection*{Exponential Map \(\exp: \mathfrak{se}(3) \to SE(3)\)}
\[
T(\xi) = \exp(\xi^\wedge) =
\begin{bmatrix}
R(\phi) & J(\phi)\rho \\
0 & 1
\end{bmatrix},
\qquad R(\phi) = \exp(\phi^\wedge)
\]
where \(J(\phi)\) is the \textbf{left Jacobian} of \(SO(3)\).

\subsection*{Jacobian Definition and Closed Form}
\[
J(\phi) = \sum_{n=0}^\infty \frac{1}{(n+1)!}(\phi^\wedge)^n =
I + \frac{1 - \cos\theta}{\theta^2}\phi^\wedge +
\frac{\theta - \sin\theta}{\theta^3}(\phi^\wedge)^2,
\qquad \theta = \|\phi\|
\]

\subsection*{Small-Angle Approximations}
For \(\theta \to 0\):
\[
R \approx I + \phi^\wedge, \qquad
J \approx I + \tfrac{1}{2}\phi^\wedge
\]
\[
\exp(\xi^\wedge) \approx
\begin{bmatrix}
I + \phi^\wedge & \rho + \tfrac{1}{2}\phi^\wedge\rho \\ 0 & 1
\end{bmatrix}
\]

% =====================================================
\end{document}
