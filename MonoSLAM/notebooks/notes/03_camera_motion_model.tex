% 03_camera_motion_model.tex
\documentclass[11pt,a4paper]{article}

% preamble.tex
\usepackage[margin=1in]{geometry}
\usepackage{amsmath, amssymb, amsfonts, bm}
\usepackage{graphicx}
\usepackage{physics}
\usepackage{hyperref}
\usepackage{enumitem}
\usepackage{titlesec}

\setlength{\parindent}{0pt}
\setlength{\parskip}{6pt}

\author{Sami Osborn}
\date{2025}

\title{Camera Motion Model}
\author{Sami Osborn}
\date{}

\begin{document}
\maketitle

% ===
\section{Purpose of the Motion Model}
% ===

We model the camera as a rigid body with position \( r_t \), velocity \( v_t \), and orientation \( q_t \).

The state is:
\[
x_t =
\begin{bmatrix}
r_t \\[1mm]
v_t \\[1mm]
q_t
\end{bmatrix},
\qquad
r_t,v_t \in \mathbb{R}^3,\;\; q_t \in \mathbb{H},\; \|q_t\| = 1
\]

Where for any point \(p_C\) expressed in camera coordinates, its world coordinates are
\[
p_W = r_t + R(q_t) p_C
\]

Here, \(R(q_t) \in SO(3)\) is the rotation matrix associated with \(q_t\).

We assume the camera moves with approximately constant velocity and constant angular velocity over each small time interval.

% ===
\section{Continuous-Time Dynamics}
% ===

Let \( \omega_t \in \mathbb{R}^3 \) denote the angular velocity expressed in the camera frame.
Let \( g \in \mathbb{R}^3 \) be the gravity vector expressed in the world frame.

The first derivative equations are
\begin{align}
\dot r_t &= v_t, \label{eq:r_dot}\\[1mm]
\dot v_t &= R(q_t) g + w_a(t), \label{eq:v_dot}\\[1mm]
\dot{q}_t &= \tfrac{1}{2} \Omega(\omega_t) q_t + w_\omega(t) \label{eq:q_dot}
\end{align}

Where
\[
\Omega(\omega) =
\begin{bmatrix}
0 & -\omega^\top \\
\omega & -[\omega]_\times
\end{bmatrix}
\]

Process noises \( w_a(t), w_\omega(t) \) are assumed white Gaussian.

% ---
\subsection*{Proof of the Quaternion Kinematic Equation}
% ---

Let a vector \(p_C\) in the camera frame be expressed in the world frame as
\(
p_W(t) = R(t) p_C
\)

If the camera rotates with angular velocity \( \omega_t \) expressed in camera coordinates, then classical rigid-body kinematics state that
\[
\dot p_W(t) = \omega_t \times p_W(t) = [\omega_t]_\times p_W(t)
\]
Differentiating \(p_W(t) = R(t)p_C\) gives
\[
\dot p_W(t) = \dot R(t)p_C = [\omega_t]_\times R(t)p_C
\]

So
\[
\boxed{\;\dot R(t) = R(t)[\omega_t]_\times\;}
\tag{1}
\]

Next, represent a unit quaternion as
\[
q =
\begin{bmatrix}
q_0 \\ \mathbf{q}
\end{bmatrix},
\qquad
\|q\|=1
\]
A small rotation of angle \( \|\omega_t\|\Delta t \) about axis \( \hat\omega_t = \omega_t/\|\omega_t\| \) corresponds to the quaternion
\[
\delta q =
\begin{bmatrix}
1 \\[1mm] \tfrac12 \omega_t \Delta t
\end{bmatrix}
+ \mathcal{O}(\Delta t^2)
\]
The quaternion update over a small interval is
\[
q(t+\Delta t) = \delta q \otimes q(t)
\]
Compute using the quaternion product formula:
\[
\delta q \otimes q
=
\begin{bmatrix}
1 \\[1mm]
\tfrac12 \omega\Delta t
\end{bmatrix}
\otimes
\begin{bmatrix}
q_0 \\ \mathbf{q}
\end{bmatrix}
=
\begin{bmatrix}
q_0 - \tfrac12 \omega^\top \mathbf{q}\, \Delta t \\[2mm]
\mathbf{q} + \tfrac12 (q_0\omega + \omega\times\mathbf{q})\,\Delta t
\end{bmatrix}
\]
Subtract \(q(t)\), divide by \(\Delta t\), and take \(\Delta t\to 0\):
\[
\dot q =
\frac12
\begin{bmatrix}
0 & -\omega^\top \\
\omega & -[\omega]_\times
\end{bmatrix}
= \frac{1}{2} \Omega q
\]

Thus we obtain the quaternion ODE above.

% ====
\section{Discrete-Time State Transition}
% ====

For timestep \( \Delta t \), assume velocity and angular velocity are constant on each interval.

% ---
\subsection{Position and Velocity Updates}
% ---

Since \( \dot r = v \) and \( \dot v = R(q)g \),
\[
v_t = v_{t-1} + R(q_{t-1}) g \,\Delta t
\]
and integrating velocity:
\[
r_t
= r_{t-1}
+ v_{t-1}\Delta t
+ \tfrac12 R(q_{t-1}) g\, \Delta t^2
\]

% ---
\subsection{Orientation Update}
% ---

Integrating \eqref{eq:q_dot} over \(\Delta t\) with constant \(\omega_{t-1}\) yields
\[
q_t
= q_{t-1} \otimes
\exp\!\left( \tfrac12\, \omega_{t-1}\Delta t \right)
\]
Where the exponential map on \(SO(3)\) is
\[
\exp(\phi) =
\begin{bmatrix}
\cos\!\left(\tfrac{\|\phi\|}{2}\right) \\[1mm]
\frac{\phi}{\|\phi\|} \sin\!\left(\tfrac{\|\phi\|}{2}\right)
\end{bmatrix}
\]

% ====
\section{Linearisation and Jacobian}
% ====

Let the error-state be
\[
\delta x =
\begin{bmatrix}
\delta r \\ \delta v \\ \delta\theta
\end{bmatrix}
\]
Where \(\delta\theta \in \mathbb{R}^3\) is the small rotation vector:
\[
q = \hat q \otimes
\begin{bmatrix}
1 \\ \tfrac12 \delta\theta
\end{bmatrix}
\]

The linearised dynamics are
\[
\delta x_t = F_t\,\delta x_{t-1}
\]

% ---
\subsection{Proof of Jacobian Blocks}
% ---

\paragraph{Position block}
\[
r_t = r_{t-1} + v_{t-1}\Delta t + \tfrac12 R(q_{t-1}) g\,\Delta t^2
\]

Clearly
\[
\frac{\partial r_t}{\partial r_{t-1}} = I_3,
\qquad
\frac{\partial r_t}{\partial v_{t-1}} = I_3\,\Delta t
\]

For the orientation dependence, use the first-order expansion
\[
R(q_{t-1}) \approx R(\hat q_{t-1}) (I + [\delta\theta]_\times)
\]

Multiplying by \(g\) gives
\[
R(q_{t-1})g
\approx
R(\hat q_{t-1})g
\;+\;
R(\hat q_{t-1}) \big([\delta\theta]_\times g\big)
\]

Since \( [\delta\theta]_\times g = \delta\theta \times g \), and using the identity
\[
\delta\theta \times g = -\, g \times \delta\theta = -\, [g]_\times \delta\theta,
\]

We obtain the perturbation
\[
\delta\!\left(R(q_{t-1})g\right)
= -\, R(\hat q_{t-1}) [g]_\times \, \delta\theta
\]

Thus the derivative of the acceleration term is
\[
\frac{\partial}{\partial \delta\theta_{t-1}}
\left(\tfrac12 R(q_{t-1}) g\, \Delta t^2\right)
=
\tfrac12 \Delta t^2
\left( -\, R(q_{t-1}) [g]_\times \right)
\]

Hence
\[
\frac{\partial r_t}{\partial \delta\theta_{t-1}}
= -\tfrac12 R(q_{t-1}) [g]_\times \Delta t^2
\]

\paragraph{Velocity block}
\[
v_t = v_{t-1} + R(q_{t-1}) g\,\Delta t
\]

Thus
\[
\frac{\partial v_t}{\partial v_{t-1}} = I_3,
\qquad
\frac{\partial v_t}{\partial r_{t-1}} = 0
\]
The orientation dependence follows the same expansion as above:
\[
\frac{\partial v_t}{\partial \delta\theta_{t-1}}
= - R(q_{t-1}) [g]_\times \Delta t
\]

\paragraph{Orientation block}

Small orientation errors evolve according to the first-order Baker-Campbell-Hausdorff (BCH) expansion of the quaternion increment
\[
\delta\theta_t
= (I_3 - [\omega_{t-1}]_\times \Delta t)\,\delta\theta_{t-1}
\]

Thus
\[
\frac{\partial\,\delta\theta_t}{\partial\,\delta\theta_{t-1}}
= I_3 - [\omega_{t-1}]_\times \Delta t
\]

\bigskip

Combining all blocks:
\[
F_t =
\begin{bmatrix}
I_3 & I_3 \Delta t &
-\tfrac12 R(q_{t-1})[g]_\times \Delta t^2 \\[2mm]
0_{3\times3} & I_3 &
-R(q_{t-1})[g]_\times \Delta t \\[2mm]
0_{3\times3} & 0_{3\times3} &
I_3 - [\omega_{t-1}]_\times \Delta t
\end{bmatrix}
\]

% ====
\section{Process Noise Propagation}
% ====

The process noise covariance \( Q_t \) models uncertainty due to unmodelled accelerations and angular velocity noise:
\[
w_t =
\begin{bmatrix}
w_a \\[1mm]
w_\omega
\end{bmatrix},
\qquad
Q_t =
\begin{bmatrix}
Q_a & 0 \\[1mm]
0 & Q_\omega
\end{bmatrix}
\]

Typical choices:
\[
Q_a = \sigma_a^2
\begin{bmatrix}
\frac{\Delta t^4}{4} I_3 & \frac{\Delta t^3}{2} I_3 \\[2mm]
\frac{\Delta t^3}{2} I_3 & \Delta t^2 I_3
\end{bmatrix},
\qquad
Q_\omega = \sigma_\omega^2 \Delta t^2 I_3
\]

The predicted covariance is then:
\[
P_t^- = F_t P_{t-1} F_t^\top + Q_t
\]
Where \( P_t^- \) is the prior covariance before incorporating observations.

% ====
\section{Manifold Formulation on SE(3)}
% ====

On the \( SE(3) \) manifold, the state can be represented as a homogeneous transform:
\[
T_t =
\begin{bmatrix}
R_t & r_t \\[1mm]
0 & 1
\end{bmatrix}
\]
The update is naturally expressed via the exponential map:
\[
T_t = T_{t-1} \exp\!\big( \hat{\xi}_t \big)
\]
Where \( \xi_t = [v_t^\top, \omega_t^\top]^\top \in \mathfrak{se}(3) \) is the twist vector, and
\[
\exp(\hat{\xi}_t) =
\begin{bmatrix}
\exp([\omega_t]_\times \Delta t) & J(\omega_t) v_t \Delta t \\[1mm]
0 & 1
\end{bmatrix}
\]
With \( J(\omega_t) \) the left Jacobian of \( SO(3) \).

% ====
\end{document}
% ====
