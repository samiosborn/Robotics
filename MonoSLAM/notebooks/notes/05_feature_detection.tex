% notebooks/notes/05_feature_detection.tex
\documentclass[11pt]{article}

% preamble.tex
\usepackage[margin=1in]{geometry}
\usepackage{amsmath, amssymb, amsfonts, bm}
\usepackage{graphicx}
\usepackage{physics}
\usepackage{hyperref}
\usepackage{enumitem}
\usepackage{titlesec}

\setlength{\parindent}{0pt}
\setlength{\parskip}{6pt}

\author{Sami Osborn}
\date{2025}


\title{Feature Detection for SLAM}
\date{January 2026}

\begin{document}
\maketitle

\tableofcontents

% ====
\section{Notation, Conventions, and Basic Models}
% ====

\subsection{Discrete image model and coordinates}

\begin{definition}[Image as a discrete signal]
An image is a function
\[
\img : \Omega \subset \Z^2 \to \R
\]
Mapping pixel coordinates $x=(u,v)\in\Omega$ to an intensity value $\img(x)$.
In practice, $\img$ may be single-channel (grayscale) or a vector-valued map (colour); in these notes we will work with a scalar field.
\end{definition}

\begin{remark}
Although $\img$ is discrete, most derivations become cleaner by passing to a smooth interpolation (e.g.\ bilinear) and treating $\img$ locally as a differentiable function on $\R^2$. We will state explicitly when we make this transition.
\end{remark}

\subsection{Local patches and windows}

\begin{definition}[Patch / window]
Let $x\in\Omega$ and let $\mathcal{W}\subset\Z^2$ be a finite, symmetric index set (e.g.\ a square).

The patch of $\img$ centred at $x$ over $\mathcal{W}$ is the collection
\[
\{\img(x+r) : r\in\mathcal{W}\}
\]
We will write sums over a window as: $\sum_{r\in\mathcal{W}}$
\end{definition}

\subsection{Interpolation and gradients}

\begin{assumption}[Differentiable local model]
When deriving gradient-based identities, we assume $\img$ is extended to a differentiable function $\tilde{\img}:\R^2\to\R$ via an interpolation operator (e.g.\ bilinear or bicubic), and write $\img$ for $\tilde{\img}$ by assumption.
\end{assumption}

\begin{definition}[Spatial gradient]
The (continuous) spatial gradient of $\img$ at $x\in\R^2$ is
\[
\grad \img(x) \coloneqq
\begin{bmatrix}
\partial_u \img(x)\\
\partial_v \img(x)
\end{bmatrix}\in\R^2
\]
In discrete implementations, $\partial_u,\partial_v$ are approximated by finite differences or derivative-of-Gaussian filters.
\end{definition}

\subsection{Photometric assumptions for tracking}

\begin{assumption}[Brightness constancy]
Let $x_t$ denote a physical scene point projected at time $t$ to image coordinate $x_t\in\R^2$. Over a short time interval, the observed intensity is approximately constant:
\[
\img_t(x_t)\approx \img_{t+\Delta t}(x_{t+\Delta t})
\]
\end{assumption}

\begin{remark}
Brightness constancy is violated by exposure changes, specularities, motion blur, rolling shutter, for example. 
For MonoSLAM-style front-ends, it remains an effective local model when combined with robust feature selection and outlier rejection.
\end{remark}

\subsection{Small-displacement linearisation}

\begin{proposition}[First-order Taylor approximation under translation]\label{prop:taylor-translation}
Let $\img:\R^2\to\R$ be differentiable. For a fixed point $x\in\R^2$ and a small displacement $\delta\in\R^2$ we have 
\[
\img(x+\delta) \;=\; \img(x) + \grad \img(x)^\T \delta \;+\; O(\norm{\delta}^2)
\]
\end{proposition}

\begin{proof}
This is the standard multivariate Taylor expansion of $\img$ about $x$:
\[
\img(x+\delta)=\img(x)+D\img(x)[\delta]+\frac{1}{2}D^2\img(x)[\delta,\delta]+\cdots
\]
Where the first differential satisfies $D\img(x)[\delta]=\grad \img(x)^\T \delta$ and the remainder is $O(\norm{\delta}^2)$ as $\norm{\delta}\to 0$
\end{proof}

% ====
\section{Image Formation and Pre-processing}
% ====

\subsection{A simple acquisition model}

\begin{assumption}[Additive noise model]
We model an observed image $\img$ as
\[
\img(x) = \img^\star(x) + \eta(x)
\]
Where $\img^\star$ is an ideal noise-free image, and $\eta$ is random noise with $\mathbb{E}[\eta(x)]=0$ and $\mathrm{Var}(\eta(x))=\sigma^2$.
\end{assumption}

\begin{remark}
This is not a complete model of camera noise (i.e. image shot noise is signal-dependent, and read noise is not i.i.d. Gaussian), but it's a sufficient approximation to justify smoothing and to quantify variance reduction.
\end{remark}

\subsection{Smoothing as convolution}

\begin{definition}[Discrete convolution with a kernel]
Let $k:\Z^2\to\R$ be a finite-support kernel. The convolution $(k * \img)$ is defined by
\[
(k * \img)(x) \coloneqq \sum_{r\in\Z^2} k(r)\,\img(x-r)
\]
Where the sum is effectively over the support of $k$. 
\end{definition}

\begin{definition}[Gaussian kernel]
For $\sigma>0$, the (continuous) isotropic Gaussian is
\[
G_\sigma(\xi) \coloneqq \frac{1}{2\pi\sigma^2}\exp\!\Big(-\frac{\norm{\xi}^2}{2\sigma^2}\Big),\qquad \xi\in\R^2
\]

Isotropic means rotation-invariant: $G_\sigma(\xi)$ depends only on the radius $\|\xi\|$, so its level sets are circles.
\end{definition}

\begin{remark}
A discrete kernel is formed by sampling $G_\sigma$ on $\Z^2$ over a finite window (e.g.\ $\{-r,\dots,r\}^2$) and renormalising the sampled weights to sum to $1$, so that convolution preserves average intensity.
\end{remark}

\begin{proposition}[Gaussian separability]\label{prop:gaussian-separable}
The 2D Gaussian factorises into a product of 1D Gaussians:
\[
G_\sigma(u,v) = g_\sigma(u)\,g_\sigma(v),
\qquad
g_\sigma(t)\coloneqq \frac{1}{\sqrt{2\pi}\sigma}\exp\!\Big(-\frac{t^2}{2\sigma^2}\Big)
\]
\end{proposition}

\begin{proof}
By direct algebra,
\[
G_\sigma(u,v)
=\frac{1}{2\pi\sigma^2}\exp\!\Big(-\frac{u^2+v^2}{2\sigma^2}\Big)
=\Big(\frac{1}{\sqrt{2\pi}\sigma}\exp\!\Big(-\frac{u^2}{2\sigma^2}\Big)\Big)
 \Big(\frac{1}{\sqrt{2\pi}\sigma}\exp\!\Big(-\frac{v^2}{2\sigma^2}\Big)\Big)
=g_\sigma(u)\,g_\sigma(v)
\]
\end{proof}

\begin{remark}
For convolution, Fubini's theorem yields $(G_\sigma * \img) = g_\sigma * (g_\sigma * \img)$, where the inner convolution is along one axis and the outer along the other. 
\end{remark}

\subsection{Variance reduction under smoothing}

\begin{proposition}[Smoothing reduces noise variance]\label{prop:variance-reduction}
Assume $\img=\img^\star+\eta$ and consider a linear filter with kernel $k$ satisfying $\sum_{r} k(r)=1$.
If $\{\eta(x)\}$ are independent with $\mathbb{E}[\eta(x)]=0$ and $\mathrm{Var}(\eta(x))=\sigma^2$, then the filtered noise
\[
\eta_k(x)\coloneqq (k*\eta)(x) = \sum_r k(r)\,\eta(x-r)
\]
satisfies
\[
\mathbb{E}[\eta_k(x)] = 0,
\qquad
\mathrm{Var}(\eta_k(x)) = \sigma^2 \sum_r k(r)^2 \;\le\; \sigma^2
\]
\end{proposition}

\begin{proof}
Linearity implies $\mathbb{E}[\eta_k(x)] = \sum_r k(r)\mathbb{E}[\eta(x-r)] = 0$

Independence gives
\[
\mathrm{Var}(\eta_k(x)) = \sum_r k(r)^2\,\mathrm{Var}(\eta(x-r)) = \sigma^2\sum_r k(r)^2
\]
Since $\sum_r k(r)=1$, the Jensen/Cauchy--Schwarz inequality implies $\sum_r k(r)^2 \le (\sum_r |k(r)|)^2 = 1$ when $k(r)\ge 0$ (as for Gaussian kernels), hence $\mathrm{Var}(\eta_k(x))\le \sigma^2$
\end{proof}

\begin{remark}
Variance reduction is only one part of the story: smoothing also attenuates high-frequency structure, which can degrade localisation.
MonoSLAM front-ends therefore smooth mildly (often implicitly via derivative-of-Gaussian gradients) rather than aggressively.
\end{remark}

\subsection{Aliasing and blur-before-subsample}

\begin{definition}[Downsampling by an integer factor]
Let $s\in\{2,3,\dots\}$. The downsampled image $\img^{\downarrow s}$ is defined on the grid $\Omega^{\downarrow s}$ by
\[
\img^{\downarrow s}(x) \coloneqq \img(sx),
\qquad x\in\Omega^{\downarrow s}\subset\Z^2
\]
\end{definition}

\begin{proposition}[Anti-aliasing (pre-blur)]\label{prop:anti-alias}
Downsampling by factor $s$ folds high-frequency content into low frequencies (aliasing). Applying a low-pass filter (e.g.\ Gaussian smoothing) with cutoff adapted to $s$ before downsampling reduces aliasing artefacts.
\end{proposition}

\begin{proof}
A complete proof uses discrete-time Fourier analysis: downsampling corresponds to periodic replication and overlap in the frequency domain. When the original spectrum is not band-limited to the Nyquist region of the downsampled grid, these replications overlap, producing aliasing. Pre-filtering attenuates energy outside the admissible band, reducing overlap. 
\end{proof}

\begin{remark}
High-frequency components are regions where intensity changes significantly. For example, edges, corners, and fine texture. 
\end{remark}

\subsection{Gaussian pyramids}

\begin{definition}[Gaussian pyramid]
Fix a smoothing kernel $G_{\sigma}$ and a downsampling factor $s=2$. Define
\[
\img^{(0)} \coloneqq \img,
\qquad
\img^{(\ell+1)} \coloneqq \Big(G_{\sigma} * \img^{(\ell)}\Big)^{\downarrow 2},
\qquad \ell=0,1,2,\dots
\]
The sequence $\{\img^{(\ell)}\}$ is called a Gaussian pyramid.
\end{definition}

\begin{remark}
For tracking (Lucas--Kanade / KLT), pyramids enable coarse-to-fine optimisation, effectively enlarging the basin of convergence for small-motion linearisations such as Proposition~\ref{prop:taylor-translation}.
\end{remark}

% ====
\section{Feature Detection}
% ===

\subsection{Repeatability, localisation, and trackability}

\begin{definition}[Repeatability (operational)]
A detector is repeatable if, under admissible changes in viewpoint and photometry, it selects corresponding physical points across frames with high probability.
\end{definition}

\begin{definition}[Localisation accuracy]
A feature is well-localised if its estimated position $\hat{x}$ has small uncertainty (e.g.\ subpixel) and small bias relative to the true image position.
\end{definition}

\begin{definition}[Trackability]
A feature at location $x$ is trackable if small inter-frame motion can be estimated reliably from local image content (typically from a patch around $x$).
\end{definition}

\subsection{Lucas--Kanade optical flow as local least squares}\label{sec:lk}

\begin{assumption}[Brightness constancy on a window]\label{ass:bc-window}
Let $W$ be a finite window of pixels around $x$. Assume that for a small displacement $u\in\R^2$,
\[
\img(p + u) \approx \img(p), \qquad \forall p\in W
\]
\end{assumption}

\begin{definition}[Lucas--Kanade local displacement estimate]\label{def:lk}
Under Assumption~\ref{ass:bc-window} and the first-order Taylor approximation
(Proposition~\ref{prop:taylor-translation}), define the residual at pixel $p\in W$ by
\[
r_p(u) \coloneqq \img(p+u) - \img(p) \approx \grad \img(p)^\T u 
\]
The Lucas--Kanade estimate is the least-squares minimiser
\[
\hat u \in \arg\min_{u\in\R^2} \sum_{p\in W} \big(\grad \img(p)^\T u\big)^2
\]
\end{definition}

\begin{proposition}[Normal equations and the structure tensor]\label{prop:lk-normal}
The minimiser in Definition~\ref{def:lk} satisfies the normal equations
\[
G\,u = 0,\qquad 
G \coloneqq \sum_{p\in W} \grad \img(p)\,\grad \img(p)^\T \in \R^{2\times 2}
\]
More generally, in the presence of a temporal term $\img_t$ (optical flow),
\[
G\,u = b,\qquad 
b \coloneqq -\sum_{p\in W} \grad \img(p)\,\img_t(p)
\]
\end{proposition}

\subsection{Aperture problem and conditioning}

\begin{proposition}[Edge ambiguity (aperture problem, local statement)]\label{prop:aperture}
Suppose that in a neighbourhood of $x$, the image intensity varies only along a single direction, i.e.\ $\grad \img(x)$ is nonzero but nearly constant and rank-1 information dominates. Then, using only local brightness constancy over a small window, the displacement component orthogonal to $\grad \img(x)$ is weakly constrained, leading to ambiguous motion estimates.
\end{proposition}

\begin{proof}
Under brightness constancy and the first-order approximation (Proposition~\ref{prop:taylor-translation}), the change in intensity for a small displacement $\delta$ is approximately $\grad \img(x)^\T \delta$.
If local gradients within a window span essentially a 1D subspace, then the data term depends primarily on the projection of $\delta$ onto that subspace, leaving the orthogonal component poorly observed. Formally, the normal equations in Lucas--Kanade involve a matrix of summed outer products of gradients; when these are nearly rank-1, the system is ill-conditioned and the solution is not stable to noise.
\end{proof}

\subsection{The second-moment matrix (structure tensor) viewpoint}

\begin{definition}[Second-moment matrix / structure tensor]
Given a window $\mathcal{W}$ and nonnegative weights $w(r)$, define the structure tensor
\[
M(x) \coloneqq \sum_{r\in\mathcal{W}} w(r)\,
\grad \img(x+r)\,\grad \img(x+r)^\T
\;\in\; \R^{2\times 2}
\]
\end{definition}

\begin{proposition}[Interpretation via eigenvalues]\label{prop:eigs-interpretation}
Let $\lambda_1(x)\ge \lambda_2(x)\ge 0$ denote the eigenvalues of $M(x)$.
Then:
\begin{itemize}
\item If $\lambda_1\approx \lambda_2 \approx 0$, the region is locally flat (little texture).
\item If $\lambda_1 \gg \lambda_2 \approx 0$, the region resembles an edge (1D texture).
\item If $\lambda_1$ and $\lambda_2$ are both large, the region is corner-like (2D texture), and local motion estimation is well-conditioned.
\end{itemize}
\end{proposition}

\begin{proof}
Each term $\grad \img , \grad \img^\T$ is positive semidefinite, so $M(x)$ is positive semidefinite and has nonnegative eigenvalues.
If all gradients are near zero within the window, then $M(x)\approx 0$ and both eigenvalues are small.
If gradients are nonzero but all aligned with a single direction $g$, then $M(x)\approx c\,gg^\T$ for some $c>0$, which is rank-1, hence one eigenvalue is large and the other near zero.
If gradients span two independent directions with substantial energy, then $M(x)$ is full-rank and both eigenvalues are large.
\end{proof}

\begin{remark}[SLAM relevance]
In MonoSLAM-style EKF updates or bundle adjustment, poorly conditioned features (small $\lambda_2$) behave like ``sliding'' constraints and degrade robustness. Hence a practical policy is to favour points with large minimum eigenvalue (Shi--Tomasi criterion), subject to spatial distribution constraints - like corners.
\end{remark}

\end{document}
