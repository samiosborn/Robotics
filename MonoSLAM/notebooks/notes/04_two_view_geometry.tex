% 04_two_view_geometry.tex
\documentclass[11pt,a4paper]{article}

% preamble.tex
\usepackage[margin=1in]{geometry}
\usepackage{amsmath, amssymb, amsfonts, bm}
\usepackage{graphicx}
\usepackage{physics}
\usepackage{hyperref}
\usepackage{enumitem}
\usepackage{titlesec}

\setlength{\parindent}{0pt}
\setlength{\parskip}{6pt}

\author{Sami Osborn}
\date{2025}

\title{Two-View Geometry}

\begin{document}
\maketitle

\section{Overview}

This document develops the mathematical foundations of two-view geometry.

Given two images of a static scene, our goals are:
\begin{itemize}
  \item Estimate the fundamental matrix
  \item Recover the relative camera pose
  \item Triangulate 3D points
\end{itemize}

\section{Camera Intrinsics and Extrinsics}

A pinhole camera maps 3D points in the world to 2D image coordinates via a projective transformation.
This mapping is parameterised by intrinsic and extrinsic components.

\subsection{Intrinsic Matrix}

The intrinsic matrix
\(
K \in \mathbb{R}^{3 \times 3}
\)
encodes the internal geometry of the camera, mapping camera-frame coordinates to pixel coordinates.

It is typically written as
\[
K =
\begin{bmatrix}
f_x & s   & c_x \\
0   & f_y & c_y \\
0   & 0   & 1
\end{bmatrix}
\]
where:
\begin{itemize}
  \item \(f_x, f_y\) are the focal lengths measured in pixels
  \item \(s\) is the skew parameter (usually zero)
  \item \((c_x, c_y)\) is the principal point
\end{itemize}

When working with normalised image coordinates, the effect of \(K\) is removed by pre-multiplying image points by \(K^{-1}\).

\subsection{Rotation}

The rotation matrix
\(
R \in SO(3)
\)
describes the orientation of the camera frame relative to the world frame.

It satisfies the defining properties:
\[
R^\top R = I, \quad \det(R) = 1
\]

Geometrically, \(R\) maps vectors from the world coordinate frame into the camera coordinate frame.

\subsection{Translation}

The translation vector
\(
t \in \mathbb{R}^3
\)
represents the position of the camera centre expressed in the camera coordinate frame.

Together with the rotation, it defines the rigid-body transformation between frames.

\subsection{Camera Projection Matrix}

The full camera projection matrix is \(P \in \mathbb{R}^{3 \times 4}\) given by
\[
P = K \,[\, R \mid t \,]
\]

A 3D point in homogeneous coordinates,
\[
X = (X, Y, Z, 1)^\top \in \mathbb{P}^3
\]
projects to an image point
\[
x \sim P X, \quad x \in \mathbb{P}^2
\]

\section{Two Views and Epipolar Geometry}

Consider two cameras observing the same 3D point \(X\):
\[
x_1 \sim P_1 X, \quad x_2 \sim P_2 X
\]

The camera centres and \(X\) define a plane in 3D known as the \emph{epipolar plane}.
This plane intersects each image plane in a line, called an \emph{epipolar line}.

The key geometric consequence is:

\begin{quote}
Given a point \(x_1\) in the first image, its correspondence \(x_2\) in the second image must lie on a specific line.
\end{quote}

This constraint is purely projective.

\section{The Fundamental Matrix}

The epipolar constraint can be written algebraically as
\[
x_2^\top F x_1 = 0,
\]
where \(F \in \mathbb{R}^{3 \times 3}\) is the \emph{fundamental matrix}.

\subsection{Properties}

The fundamental matrix satisfies:
\begin{itemize}
  \item \(F\) is defined up to scale
  \item \(\mathrm{rank}(F) = 2\)
  \item it maps points to epipolar lines: \(l_2 = F x_1\)
\end{itemize}

Geometrically, the constraint enforces that the vectors \(x_2\), \(F x_1\), and the origin are coplanar.

\section{Geometric Interpretation}

In the calibrated case, the fundamental matrix factors as
\[
F = [t]_\times R,
\]
where
\[
[t]_\times =
\begin{bmatrix}
0 & -t_z & t_y \\
t_z & 0 & -t_x \\
-t_y & t_x & 0
\end{bmatrix}
\]
is the skew-symmetric matrix representing the cross product.

The epipolar constraint
\[
x_2^\top [t]_\times R x_1 = 0
\]
expresses the fact that the two viewing rays and the translation vector lie in the same plane.

\section{Estimating the Fundamental Matrix}

Each point correspondence \((x_1, x_2)\) yields one linear constraint:
\[
x_2^\top F x_1 = 0.
\]

Expanding this equation produces a linear system
\[
A f = 0,
\]
where \(f = \mathrm{vec}(F)\).

With at least eight correspondences, the solution is obtained via Singular Value Decomposition (SVD).

\subsection{Rank-2 Enforcement}

The estimated matrix must satisfy \(\det(F) = 0\).
This is enforced by:
\begin{enumerate}
  \item computing the SVD \(F = U \Sigma V^\top\)
  \item setting the smallest singular value to zero
  \item reconstructing \(F\)
\end{enumerate}

This step restores the correct geometric structure.

\section{From Fundamental to Essential Matrix}

If the cameras are calibrated, we define the \emph{essential matrix}
\[
E = K^\top F K.
\]

The essential matrix satisfies:
\begin{itemize}
  \item \(\mathrm{rank}(E) = 2\)
  \item two equal nonzero singular values
\end{itemize}

It admits the factorisation
\[
E = [t]_\times R,
\]
from which the relative camera pose can be recovered.

\section{Recovering Camera Pose}

The SVD of \(E\) yields four possible solutions for \((R, t)\).
This ambiguity arises from sign and reflection symmetries.

The physically valid solution is selected using the \emph{cheirality condition}:
\begin{quote}
Triangulated points must lie in front of both cameras.
\end{quote}

Only one candidate satisfies this constraint.

\section{Triangulation}

Given camera matrices \(P_1, P_2\) and corresponding image points \(x_1, x_2\),
the 3D point \(X\) is recovered by solving
\[
x_i \times (P_i X) = 0, \quad i = 1,2.
\]

Stacking the resulting equations yields a homogeneous linear system
\[
A X = 0,
\]
which is solved via SVD.
The solution is defined up to scale.

\section{What This Enables}

Two-view geometry provides the foundation for:
\begin{itemize}
  \item Structure-from-Motion
  \item Visual odometry
  \item Visual SLAM
  \item Multi-view reconstruction
\end{itemize}

All subsequent methods refine or extend the principles developed here.

\section{Key Takeaway}

Two-view geometry eliminates the 3D scene to recover camera geometry,
then reintroduces 3D structure via triangulation.

It is the minimal, exact, and geometric core of visual reconstruction.

\end{document}
